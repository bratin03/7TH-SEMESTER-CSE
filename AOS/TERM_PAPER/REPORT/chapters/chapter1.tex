% \chapter{Introduction}

% \label{Chapter1} % Reference label

% \lhead{Chapter 1. \emph{Introduction}} % Header for each page


\section{Introduction}

Google operates one of the largest and most complex infrastructures in the world, with millions of servers across data centers worldwide. Monitoring and managing performance metrics across this massive infrastructure requires a database capable of handling petabytes of data and millions of queries per second.

Google uses various types of databases, applications each suited for specific tasks. There should be monitoring system over all these systems. Generally a TSDB (Time-Series Database monitoring system is used which captures events along with time stamps)


\subsection{Monarch as TSDB}

A time-series database (TSDB) is a specialized database optimized to handle, store, and analyze time-series data, which consists of data points indexed by time. This type of data typically represents how a system, process, or measurement evolves over time, with each entry or observation paired with a timestamp

At Google, the time-series database (TSDB) landscape has evolved significantly over the years, primarily driven by internal projects and open-source contributions to handle the massive scale of data generated by Google's infrastructure

The initial TSDB developed and used was Borgmon which was replaced with Monarch for reasons discussed later in the report

So currently Monarch is the globally-distributed in-memory time series database monitoring system in Google
